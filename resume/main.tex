\documentclass[a4paper]{article}
\usepackage[margin=1in]{geometry}
\usepackage[utf8]{inputenc}
\usepackage[spanish]{babel}
\usepackage{amsmath, amsthm, amsfonts, amssymb, mathrsfs}
\usepackage{hyperref,fancyhdr}
%\usepackage{booktabs}% fancy tables
%\usepackage{tocloft}% agregar tabla de preguntas 
%\usepackage{multicol}
\usepackage{graphicx}
%\usepackage{capt-of}
\setlength{\columnsep}{1cm}
\IfFileExists{lmodern.sty}{\usepackage{lmodern}}{}
\usepackage[T1]{fontenc}
\IfFileExists{lmodern.sty}{\usepackage{lmodern}}{}
\usepackage[T1]{fontenc}
\usepackage[utf8]{inputenc}
\usepackage{epstopdf}
\usepackage[section]{placeins}
\author{Carlos Rodrigo Sanabria Flores}
\hypersetup{
    pdftitle={Apuntes RC},
    pdfsubject={RC},
    pdfauthor={Carlos Rodrigo Sanbria},
    pdfkeywords={RC}
}
\title{Fución de Ackermann}
\begin{document}
\maketitle
\begin{abstract}
    Resultados obtenidos con la función de Ackermann.
\end{abstract}
\section{Resultados}
\pagestyle{fancy}
\fancyhf{}
\renewcommand{\headrulewidth}{0mm}
\cfoot{\thepage}
\setlength{\parskip}{2em}
\begin{table}[h]
    \centering
    \begin{tabular}{|c|c|c|c|c|c|}
        \hline
        \textbf{m\textbackslash n} & \textbf{0} & \textbf{1} & \textbf{2} & \textbf{3} & \textbf{4} \\ \hline
        \textbf{0} & 1 & 2 & 3 & 4 & 5 \\ \hline
        \textbf{1} & 2 & 3 & 4 & 5 & 6 \\ \hline
        \textbf{2} & 3 & 5 & 7 & 9 & 11 \\ \hline
        \textbf{3} & 5 & 13 & 29 & 61 & 125 \\ \hline
        \textbf{4} & 13 & 65533 & \textbf{$\infty$} & \textbf{$\infty$} & \textbf{$\infty$} \\ \hline
    \end{tabular}
    \caption{Resultados Ackermann}
    \label{table:Ack}
\end{table}
\end{document}